\documentclass[12pt,a4paper]{article}
\usepackage[utf8]{inputenc}
\usepackage[ngerman]{babel}
\usepackage{setspace}
\usepackage{graphicx}
\usepackage{caption}
\usepackage{geometry}
\usepackage{amsmath}
\usepackage{fancyhdr}
\usepackage[hidelinks]{hyperref}
\usepackage[numbers]{natbib}
\usepackage{setspace}
\usepackage{tocloft}
\usepackage{minted}
\usepackage{pdfpages}
\usepackage{siunitx}
\usepackage{booktabs}
\usepackage{xurl}
% \usepackage{svg}

% for German quotation marks
\usepackage{csquotes}
\MakeOuterQuote{"}

% Use German unit system
\addto\extrasngerman{\sisetup{locale = DE}}

% Seitenränder
\geometry{left=2.5cm,right=2.5cm,top=2cm,bottom=2cm} % für Jugend forscht

% Bibliographie-Kommando
\newcommand{\bib}{\bibliographystyle{plainnat}\bibliography{bibliography}}

% Kopf- und Fußzeile
\pagestyle{fancy}
\fancyhf{}
\rhead{Abschlussbericht}
% \lhead{Vorname Name}
\rfoot{\thepage}

% Abschnittsabstand
\setlength{\parskip}{0.5em}
\setlength{\parindent}{0pt}

\begin{document}
	
	% Titelseite
	\begin{titlepage}
		% \begin{minipage}[c]{0.49\textwidth}
		% 	\raggedright
		% 	\includegraphics[width=0.4\textwidth]{TH_Mannheim_Logo_RGB_blau.png}
		% \end{minipage}
		% \hfill
		% \begin{minipage}[c]{0.49\textwidth}
		% 	\raggedleft
		% 	\includegraphics[width=0.4\textwidth]{logo.png}
		% \end{minipage}
		% \\[1cm]
		\centering
		{\LARGE \textbf{Der Crawler - Ein Reinforcement-Learning-Krabbelroboter}\par}
		\vfill
		% {\Large \textbf{Thema der Arbeit}} \\[2cm]

		\includegraphics[width=\textwidth, height=0.35\textheight, keepaspectratio]{photos/crawler_deckblatt.png}
		
		\vfill
			\textbf{\large Projektteam}\\[1em]
			\begin{minipage}[t]{0.32\textwidth}
			    \centering
			    \textbf{Gregor Niehl}
			\end{minipage}
			\hfill
			\begin{minipage}[t]{0.32\textwidth}
			    \centering
			    \textbf{Jonathan Kraus}
			\end{minipage}
			\hfill
			\begin{minipage}[t]{0.32\textwidth}
			    \centering
			    \textbf{Pascal Roth}
			\end{minipage}
		\vfill

		\begin{minipage}{0.95\textwidth}
			\textbf{\large Projektübersicht}\\[0.2cm]
			\small
			\textbf{Fragestellung:} Das Projekt befasst sich mit der Optimierung eines einbeinigen Krabbelroboters, der mithilfe von künstlicher Intelligenz selbstständig Fortbewegungsmuster erlernen soll. Ziel war es, die bestehende Mechanik und Software grundlegend zu überarbeiten, um ein robustes und anschauliches Demonstrationsobjekt zu schaffen. \\ 
			\textbf{Methode:} Der Roboter wurde in Leichtbauweise neu konstruiert und auf das Framework ROS 2 portiert. Ein zentraler methodischer Schwerpunkt lag dabei in der strikten Trennung von Hardware-Ansteuerung und KI-Logik, was eine modulare Steuerung der Servo-Motoren und eine stabile Datenverarbeitung ermöglichte. \\
			\textbf{Ergebnisse:} Nach erfolgreicher Implementierung des Q-Learning-Algorithmus konnte beobachtet werden, wie der Roboter nach der Trainingsphase erste eigenständige Schritte ausführte. Ein begleitendes Webinterface visualisiert diesen Lernfortschritt nun für Außenstehende in Echtzeit. \\
			\textbf{Diskussion:} Das Projekt zeigt, dass die Kombination aus modularer Bauweise und moderner Softwarearchitektur ein stabiles System für KI-Experimente schafft. Zukünftig bietet der "Crawler" eine ideale Basis, um weitere Sensoren oder komplexere Bewegungsmodelle zu testen. \\
		\end{minipage}
		\vfill
		
		\textbf{Dokumentation Jugend Forscht 2026} \\
		% Durchgeführt an der Technischen Hochschule Mannheim \\
		% Betreuer: Prof. Dr. Thomas Ihme \\
		% \vfill
		
		\today
	\end{titlepage}
	
	% Inhaltsverzeichnis
	\tableofcontents
	\newpage

	\fontfamily{ptm}\selectfont % Times New Roman
	\fontsize{11pt}{13.2pt}\selectfont
	\onehalfspacing % 1.5 Zeilenabstand
	
	\textit{Alle Photos und Graphiken in diesem Dokument wurden, wenn nicht anders gekennzeichnet, von Mitgliedern des Teams erstellt.}

% \section*{Abstract}
% \addcontentsline{toc}{section}{Abstract}

% Zusammenfassung (in Englisch); \\ Vier Sätze: 1. worum geht es allgemein, Kontext des Problems; 2. welches Problem wird hier bearbeitet; 3. wie wird das Problem im Wesentlichen gelöst; 4. was ist das hauptsächliche Ergebnis; insg. nicht mehr als 10 Zeilen

% We continued development on a robot that learns to crawl with a two-jointed leg using reinforcement learning algorithms. Most of our work was concentrated on improving existing aspects of the project rather than adding new features. These aspects included both the mechanical design of the robot as well as the software running on it. As a result, all project components are reimplemented from scratch and improved compared to the preceding project.

% \section{Einleitung}

% Dieses Projekt wurde im Rahmen der Kooperationsphase des Hector-Seminars im Schuljahr 2024/25 durchgeführt. 
% Ziel ist es, einen bestehenden einbeinigen Krabbelroboter in softwareseitiger und mechanischer Hinsicht zu überarbeiten und als funktionierendes Demonstrationsobjekt zu realisieren.

% Die Aufgabenstellung ist dabei bewusst eher frei gehalten. Der Fokus liegt auf der Entwicklung einer steuerbaren Softwarearchitektur mit ROS und der mechanischen Neukonstruktion des Krabbelsroboters.

% Der Rest des Artikels gliedert sich wie folgt:

% \begin{itemize}
% 	\item \textbf{Grundlagen:} Einführung in vorwiegend softwareseitige Konzepte, die später vorausgesetzt werden
% 	\item \textbf{Konzeption des neuen Krabbelroboters:} Zielsetzungen und Vorgehensweisen
% 	\item \textbf{Realisierung des neuen Krabbelroboters:} Besprechung von unvorhergesehenen Hürden in der Umsetzung
% 	\item \textbf{Diskussion:} Rückblick und Bewertung der Arbeit, Ausblick für die Zukunft
% 	\item \textbf{Danksagung, Quellen, Literatur}
% \end{itemize}

\section{Fachliche Kurzfassung}
Die fachliche Zielsetzung dieses Projekts war die Realisierung eines funktionsfähigen Demonstrationsobjekts, das den Lernprozess einer Künstlichen Intelligenz physisch sichtbar macht. Im Mittelpunkt stand die Frage, wie ein einbeiniger Roboter ohne vorprogrammierte Bewegungsabläufe allein durch Reinforcement Learning eine zielgerichtete Vorwärtsbewegung erlernen kann.

Methodisch wurde das Vorgängermodell, hier Legacy-Projekt genannt, grundlegend überarbeitet. Durch den Einsatz von CAD-Design und 3D-Druck wurde eine deutliche Gewichtsreduktion erreicht, was die mechanische Belastung verringerte. Softwareseitig wurde das System auf das ROS 2 Framework portiert. Hierbei wurde eine strikte Trennung zwischen der Hardware-Steuerung und der KI-Logik umgesetzt, um eine modulare Erweiterbarkeit und eine effiziente Entwicklung im Team zu ermöglichen.

Als Lernalgorithmus wurde Q-Learning eingesetzt. Die Ergebnisse zeigen, dass der Roboter nach der Trainingsphase in der Lage ist, die beiden Gelenke seines Beins stabil zu koordinieren. Die daraus resultierende Vorwärtsbewegung ist aufgrund technischer Limitierungen und einer bewussten softwareseitigen Begrenzung zur Schonung der Servomotoren langsam, stellt jedoch eine, unter Berücksichtigung der stochastischen Natur des Lernverfahrens, reproduzierbare Lösung des Bewegungsproblems dar.

Zur Analyse des Lernverhaltens wurde ein Webinterface auf Basis von React entwickelt. Dieses visualisiert die Zustandsübergänge und die Aktualisierung der Q-Table in Echtzeit. Durch die Bereitstellung dieser Daten über eine Websocket-Verbindung wird der normalerweise abstrakte Lernprozess ("Black Box") transparent und für den Betrachter nachvollziehbar gemacht. Zusätzlich dient das Interface als zentrale Steuereinheit, über welche die Hyperparameter des Algorithmus, bspw. die Learning-Rate oder der Discount-Factor, im laufenden Betrieb angepasst werden können, was eine effiziente Durchführung systematischer Versuchsreihen ermöglicht.

\section{Motivation und Fragestellung}

Unser Team fand im Rahmen der Kooperationsphase 2025 des Hector-Seminars an der Technischen Hochschule Mannheim zusammen, da wir alle ein starkes Interesse an der Informatik teilen. Bei der Projektwahl sagte uns der Crawler sofort zu, weil er als interdisziplinäres Projekt perfekt zu unseren damaligen Interessen passte: Einer von uns wollte das CAD-Design erlerne und in den 3D-Druck einarbeiten, während der zweite sich für Roboterprogrammierung interessierte und der dritte sich für die Entwicklung eines modernen, benutzerfreundlichen Webinterfaces begeisterte. Diese natürliche Aufteilung erlaubte es uns, das Projekt als echtes Team zu realisieren, bei dem jeder seine Interessen und Stärken einbringen konnte, während auch das getrennte Arbeiten aufgrund der räumlichen Distanz problemlos möglich war. 

Der Ausgangspunkt unserer Arbeit war ein bestehendes Robotermodell eines Studentenprojekts und eines nachfolgenden Projekts im Hector-Seminar. Dieses offenbarte jedoch erhebliche mechanische Schwächen und funktionierte softwareseitig kaum. Schnell wurde uns klar, dass die ursprüngliche Zielsetzung unserer Vorgänger, der systematische Vergleich verschiedener Lernalgorithmen, in der Realität an der Hardware scheitert und nicht zielführend ist. Ein einbeiniger Roboter ohne Radantrieb hat kaum einen industriellen Nutzwert. Sein wahrer Wert liegt in der Vermittlung von Konzepten. Daher haben wir unseren Fokus verschoben: Weg von abstrakten Testreihen, hin zur Realisierung einer robusten Demonstrationsplattform. 

Unser Ziel war es zu zeigen, dass man durch durch saubere Mechanik und Softwarearchitektur ein System bauen kann, dass Reinforcement Learning nicht nur ausführt, sondern auch begreifbar macht. Unsere zentrale Fragestellung lautete daher, ob sich der \glqq Black-Box-Charakter\grqq von KI-Algorithmen durch eine direkte Kopplung von physischer Bewegung und einer interaktiven Benutzeroberfläche  auflösen lässt. Wir wollten zeigen, dass ein Webinterface, das die internen Lernwerte live abbildet, den Lernprozess transparenter macht und so das Verständnis für die Funktionsweise von Reinforcement Learning fördert.
	\section{Hintergrund und Theoretische Grundlagen}
\label{sec:grundlagen}

Im folgenden Abschnitt werden Konzepte erklärt, die für die softwareseitige Entwicklung des Krabbelroboters grundlegend waren. Diese Konzepte werden in späteren Teilen des Artikels vorausgesetzt.

\subsection{Robot Operating System (ROS)}

Das Robot Operating System (ROS)\cite{ros} ist ein quelloffenes Softwareframework zur Entwicklung von Robotern. Es bildet eine Middleware zur Verwaltung einzelner Softwarekomponenten, sowie deren Kommunikation untereinander. Diese Komponenten (genannt "Nodes") können jeweils in einer beliebigen Programmiersprache (hier Python) implementiert, und bei Programmstart von ROS ausgeführt werden. Die Nodes laufen in getrennten Prozessen. Kommunikation zwischen diesen Nodes geschieht über Kanäle in ROS (genannt "Topics"), auf denen Nachrichten von sog. Publisher-Nodes gesendet und von sog. Subscriber-Nodes gelesen werden können. Die Datentypen dieser Nachrichten können pro Topic festgelegt werden. Diese Datentypen können grundlegende Typen wie Zahlen oder Strings, aber auch komplexere, im Code festgelegte Datenstukturen sein.

\subsection{Der Lernprozess: Reinforcement Learning}
\label{sec:reinforcement_learning}

Im Maschinellen Lernen wird, um selbstlernende Algorithmen zu trainieren, meist das Konzept des Reinforcement Learning (RL) angewendet. Das Konzept besteht in der Regel aus zwei Gegenspielern:
\begin{itemize}
    \item Dem Algorithmus, der (anfangs zufällig, später gezielt) Aktionen durchführt.
    \item Einer Umgebung (dem Environment), welche die Aktionen des Algorithmus anhand einer gegebenen Statistik bewertet, und dem Algorithmus anhanddessen Feedback gibt, den sog. Reward. Diesen Reward benutzt der Algorithmus dann, um sich eigenständig zu verbessern.
\end{itemize}

In der jetzigen Implementation des Crawlers wird der Reward alleine auf Basis der Veränderung der Radposition berechnet. Wenn die Positionen der Inkrementalgeber auf eine Vorwärtsbewegung schließen lassen, erhält der Algorithmus einen positiven Reward. Bei einer Rückwärtsbewegung oder keiner Veränderung ist der Reward negativ.

\subsection{Der Lernprozess: Das Q-Learning}

Das Q-Learning ist ein Algorithmus, der auf die Auswahl einer Aktion aus einer Reihe an Möglichkeiten spezialisiert ist. Dem Algorithmus liegt die Q-Table zugrunde, eine Tabelle, in der jede Zelle eine spezifische Kombination aus einer Input-Option (einem Zustand der Umgebung) und einer Output-Option (einer Aktion) darstellt. Jede Zelle ist außerdem mit einer Zahl besetzt. Soll der Algorithmus eine Aktion auswählen, wählt er aus den Zellen, deren Input-Option dem Zustand der Umgebung entspricht, die Zelle, deren Zahl am höchsten ist. Die zugehörige Aktion ist dann das Ergebnis des Algorithmus. Wird das Q-Learning in Kombination mit dem Konzept des RL angewandt, wird die Zahl in der ausgewählten Zelle je nach resultierendem Reward angepasst: Bei positivem Reward wird sie vergrößert, bei negativem Reward verringert. Um den Einfluss verschiedener Aktionen auf die Umgebung zu erkunden, wird der Output anfangs noch ab und zu zufällig ausgewählt, mit voranschreitender Wiederholungszahl passiert dies allerdings immer seltener.

In der Q-Learning-Implementation des Crawlers werden die Positionen der Motoren an Bein- und Fuß-Gelenk des Roboters als Input-Optionen genutzt. Um die Dimensionen der Q-Table zu begrenzen, werden die Positionen allerdings nur grob als "oben", "mittig" oder "unten" ausgelesen. Output-Optionen sind "Bein hoch", "Bein runter", "Fuß hoch" und "Fuß runter".


\newcommand{\mc}[1]{\multicolumn{1}{c}{#1}}
\begin{table}
	\captionsetup{justification=centering}
	\caption{Beispielhafte Visualisierung der Q-Table beim Crawler \\ (Zahlen entsprechen keinem realen Szenario)}
	\centering
	\begin{tabular}{ llSSSS }
		\toprule
		{} & {} & \multicolumn{4}{c}{Aktionen} \\
		\cmidrule{3-6}
		Zustand Bein & Zustand Fuß & \mc{Bein hoch} & \mc{Bein runter} & \mc{Fuß hoch} & \mc{Fuß runter} \\
		\midrule
		Oben & Oben & 0,52 & 0,64 & 0,83 & 0,21 \\
		& Mittig & 0,80 & 0,06 & 0,75 & 0,66 \\
		& Unten & 0,46 & 0,85 & 0,51 & 0,67 \\
		\addlinespace
		Mittig & Oben & 0,13 & 0,88 & 0,81 & 0,10 \\
		& Mittig & 0,76 & 0,42 & 0,90 & 0,66 \\
		& Unten & 0,28 & 0,73 & 0,52 & 0,25 \\
	    \addlinespace
		Unten & Oben & 0,75 & 0,90 & 0,34 & 0,49 \\
		& Mittig & 0,93 & 0,22 & 0,26 & 0,62 \\
		& Unten & 0,89 & 0,62 & 0,59 & 0,57 \\
	    \bottomrule
	\end{tabular}
% \label{fig:example-q-table}
\end{table}

	\section{Vorgehensweise, Materialien und Methoden}
\label{sec:konzeption}

Im folgenden Abschnitt wird der Zustand des Projekts bei Übernahme sowie die daraus abgeleiteten Zielsetzungen beschrieben. Darauf aufbauend werden die gewählten technischen Methoden und Materialien erläutert.

Die Projektbearbeitung erfolgte in einem iterativen Entwicklungsprozess, aufgeteilt in die Bereiche Hardware und Software, was eine parallele Umsetzung und kontinuierliche Integrationstests ermöglichte. Zur Versionierung und Dokumentation wurde eine GitHub-Organisation genutzt, die den eigenen Anteil an der Entwicklung transparent nachvollziehbar macht.

\subsection{Mechanischer Aufbau und Konstruktionsprinzipien}

\subsubsection{Analyse des Ausgangsmodells}

Zu Beginn des Projekts wird der einbeinige Krabbelroboter des Legacy-Projekts in Bezug auf Software, Hardware und Funktionsumfang analysiert. Dazu wird besonders die Dokumentationen der Vorgängerprojekte angesehen \cite{vorgängerprojekt}. 

Der aktuelle Zustand des Roboters zum Zeitpunkt der Übernahme ist in Abbildung \ref{fig:crawler_old} dargestellt und ist für die folgenden Abschnitte relevant.

\begin{figure}[h]
  \centering
  \includegraphics[width=0.8\textwidth]{crawler_old.jpeg}
  \caption{Roboter zum Übernahmezeitpunkt}
  \label{fig:crawler_old}
\end{figure}

\textbf{Mechanische Analyse}

Der Roboter wurde dabei einer mechanischen Strukturanalyse unterzogen. Er bestand zum Zeitpunkt der Übernahme aus einer Basis (Chassis) und einem zweigliedrigen Bein, welche über zwei Servomotoren (Rotationsgelenke) als kinematische Kette verbunden sind.

% Daher konnte die Anzahl $n$ an Freiheitsgraden (DoF) des Roboters mit der Formel von Grübler berechnet werden. Ein Freiheitsgrad ist dabei \glqq jede verbleibende unabhängige Bewegungsfreiheit eines Systems\grqq \ \cite{robotik_scriptum}.
% \begin{align}
%   n = m \cdot (N - 1 - J) + \sum_{i=1}^{J}f_i \noindent \\
%   n = 6 \cdot (3 - 1 - 2) + \sum_{i=1}^{2}1 = 6 \cdot 0 + 2 = 2
% \end{align}
% Dabei ist \glqq $m$ die Anzahl der Freiheitsgrade jedes Starrkörpers ($m = 6$ im dreidimensionalen Raum)\grqq \ \cite{robotik_scriptum}, $N$ die Anzahl der Starrkörper (hier 3: Chassis, Bein, Fuß), $J$ die Anzahl der Gelenke (hier 2) und $f_i$ die Anzahl an Freiheitsgraden des Gelenks $i$. 

% Des Weiteren wurde die Form des Konfigurationsraum des Roboters untersucht. Der Konfigurationsraum beschreibt alle möglichen Positionen und Orientierungen des Roboters im Raum. Mit der Form ist seine topologische Struktur gemeint, sprich wie die möglichen Konfigurationen (alle Konfigurationen zusammen ergeben den Konfigurationsraum) zusammenhängen. Da der Krabbel-Roboter zwei zyklische Rotationsgelenke hat, ergibt sich die Form eines zweidimensionalen Torus $(T^2 = S^1 \times S^1)$, wobei $S^1$ die Menge aller Punkte auf einem Kreis darstellt \cite{robotik_scriptum}. Abbildung \ref{fig:torus} zeigt dies visuell.
% \begin{figure}[h]
%   \centering
%   \includegraphics[width=0.7\textwidth]{torus.png}
%   \caption{Der Konfigurationsraum des Krabbelroboters als Torus, aus \glqq Grundlagen der Robotik\grqq \ \cite{robotik_scriptum}.}
%   \label{fig:torus}
% \end{figure}

% Das Robotersystem entspricht also einem (autonomen) mobilen Roboter mit 2 DoF und einem planaren 2R-Roboterbein. Daraus ließe sich eine Kinematische Modellierung bzw. Simulation erstellen, die allerdings nicht im Rahmen dieser Abschlussarbeit angefertigt wurde. 

\textbf{Funktionsumfang}

Der Funktionsumfang des Roboters beschränkte sich auf die manuelle Steuerung der Motoren über eine graphische Benutzeroberfläche (GUI).

\subsubsection{CAD-gestütztes Design und 3D-Druck} % TO-DO: in "grundlagen" verschieben?

Die Bauteile und das Gehäuse wurden in einem CAD-gestützten Prozess mit \emph{Autodesk Fusion 360} \cite{fusion360} konstruiert. Die digitale Modellierung ermöglichte eine präzise Gestaltung und direkte Überprüfung der Komponenten, die für den Datenaustausch, insbesondere mit Herrn Prof. Dr. Ihme, als standardisierte STEP-Dateien exportiert wurden.

% Der 3D-Druck ermöglichte anschließend eine kostengünstige und ressourcenschonende Fertigung, da durch das additive Verfahren nur das tatsächlich benötigte Material verwendet wurde, was zudem im Vergleich zum Legacy-Projekt Gewicht spart. Dieser Prozess erlaubt eine iterative und flexible Entwicklung ohne komplexe Werkzeuge, was für den Prototypenbau ideal ist (siehe Abbildung \ref{fig:cad_3d_druck}).

Die Fertigung erfolgte mittels Fused Deposition Modeling (FDM), was eine schnelle Iteration der Prototypen ermöglichte. Abbildung \ref{fig:cad_3d_druck} zeigt das gerenderte 3D-Modell sowie ein Foto der oberen Ebene nach dem 3D-Druck.

\begin{figure}[h]
  \centering
  \includegraphics[width=0.45\textwidth]{Vorne-Render.png}
  \hspace{0.05\textwidth}
  \includegraphics[width=0.45\textwidth]{3d-druck.jpg}
  \caption{Gerendertes 3D-Modell aus Fusion 360 (links), obere Ebene nach dem 3D-Druck (rechts)}
  \label{fig:cad_3d_druck}
\end{figure}

\subsubsection{Gestaltungsschwerpunkte}
\label{subsubsec:gestaltungsschwerpunkte}

Ziel der Weiterentwicklung des übernommenen einbeinigen Krabbelroboters war es, ein modular aufgebautes, leichtes und anschaulich gestaltetes System zu schaffen, das sowohl für Demonstrations- als auch Auswertungszwecke geeignet ist. Dabei wurden folgende Anforderungen von uns anfangs festgelegt:

\begin{itemize}
  \item \textbf{Modularität (Basic):} Das Austauschen oder Erweitern zentraler Komponenten (insbesondere des Akkus) sollte erleichtert werden, etwa durch Steck- oder Schraubverbindungen.
  \item \textbf{Offenes Design:} Für Demonstrationszwecke sollten alle Komponenten sichtbar und gut zugänglich montiert sein, um einen schnellen Überblick über das System zu ermöglichen.
  \item \textbf{Verbesserter Schwerpunkt:} Damit das Bein möglichst wenig Last stemmen muss, sollte ein Schwerpunkt nah an der Radachse angestrebt werden.
  \item \textbf{Leichtbauweise:} Ein geringes Gewicht unterstützt die Energieeffizienz und erleichtert die Steuerung durch kleinere Antriebsmotoren.
\end{itemize}

Diese Zielkriterien bildeten die Grundlage für die Auswahl der Bauteile einschließlich Akku und das mechanische Design des weiterentwickelten Roboters.

\subsection{Bau des neuen Crawlers} % verschoben von unten

% Dieses Unterkapitel beschreibt die Hardware des neuen Krabbelroboters, die aus den Komponenten des Vorgängerprojekts übernommen wurde, sowie die neuen Komponenten, die für die Weiterentwicklung ausgewählt wurden. Zusätzlich wird der letztendliche mechanische Aufbau des Crawlers erläutert.
Dieses Unterkapitel beschreibt die übernommenen und neuen Hardwarekomponenten sowie den mechanischen Aufbau des Krabbelroboters.

\subsubsection{Auswahl der Hauptkomponenten}

\textbf{Der Microcontroller}

Der Prozessor des Crawlers ist ein Raspberry Pi 4 Model B, der gleiche Single Board Computer (SBC) wie beim Vorgängerprojekt.

% Zu Anfang unseres Projekts hatten wir lange Zeit Schwierigkeiten, die neuere Ubuntu-Version auf dem SBC zu installieren. Nach langem Debuggen mit verschiedenen Linux-Distributionen, Netzteilen und SD-Karten kamen wir zu dem Schluss, dass das Modell des Vorgängerprojekts zu Schaden gekommen war, seit wir es übernommen hatten, möglicherweise im Transport.

\textbf{Die Sensoren}

Zur Messung von Drehzahlunterschieden an den Rädern werden die zwei Inkrementalgeber des Typs \textit{MEC22} (PWB) \cite{pwb_me16_datasheet_2011} aus dem Legacy-Projekt weiterverwendet. Mit einer Auflösung von 500 Zählimpulsen pro Umdrehung (Counts per Revolution, CPR) und einem Raddurchmesser von \qty{5}{\centi\m} entspricht ein Impuls ca. \qty{0,31}{\milli\m} zurückgelegte Strecke. Dies ermöglicht eine präzise Erfassung kleinster Bewegungen für das Reinforcement Learning. Die Stromversorgung erfolgt direkt über den Raspberry Pi.

\textbf{Die Aktoren}

Für die Beinbewegung wurden Servomotoren vom Typ \textit{Dynamixel XL430-W250-T} (Robotis) \cite{robotis_xl430} eingesetzt. Sie bieten ein gutes Verhältnis von Drehmoment (bis 1{,}5\,Nm) zu Energieverbrauch sowie präzise Positionssteuerung und sind über eindeutige IDs adressierbar.

Die Verbindung zum Steuerrechner erfolgt via TTL-Protokoll über den \textit{U2D2-Connector} \cite{robotis_u2d2}, der per USB an den Raspberry Pi angeschlossen wird. Ein \textit{U2D2 Power Hub} \cite{robotis_u2d2_power_hub} gewährleistet dabei die stabile Spannungsversorgung aller Motoren. Vergleiche dazu auch das nachfolgende Unterkapitel zur Spannunsversorgung. 

\subsubsection{Spannungsversorgung}

Da der alte Akku, ein \textit{XCell LiPo Cracker CAR} \cite{xcell_akku} mit zwei Zellen, \qty{7.4}{\volt} und \qty{5400}{\milli\ampere\hour}, durch sein hohes Gewicht von \qty{310}{\gram} ca. \qty{25}{\percent} des Gesamtgewichts des Roboters ausmachte, wurde ein leichterer Akku gesucht.

Der neu ausgewählte \textit{Gens ace Modellbau-Akkupack (LiPo)} \cite{gens_ace_akku} ist mit seinen ebenfalls \qty{7.4}{\volt} Ausgangsspannung ähnlich aufgebaut, wiegt allerdings nur noch \qty{66}{\gram} und ist deutlich kompakter. Aufgrund des geplanten Einsatzes als Demonstrationsroboter ist auch die damit verbundene geringere Laufzeit kein Problem.

\begin{align*}
t_{\text{Betrieb, alt}} &= \frac{\qty{5400}{\milli\ampere\hour}}{\qty{1070}{\milli\ampere}} \approx \qty{5}{\hour} \\
t_{\text{Betrieb, neu}} &= \frac{\qty{1500}{\milli\ampere\hour}}{\qty{1070}{\milli\ampere}} \approx \qty{1.4}{\hour}
\end{align*}

\qty{1070}{\milli\ampere} sind dabei der maximal gemessene Strom (bei der Bewegung der Motoren) und nicht der Normalzustand.

\subsubsection{Mechanische Umsetzung} % verschoben aus "umsetzung" bzw. "ergebnisse"

Basierend auf den in Abschnitt \ref{subsubsec:gestaltungsschwerpunkte} definierten Zielen wurde die praktische Umsetzung des Chassis eingeleitet.

Zunächst erfolgte eine Recherche der Maße der Bauteile. Auf Basis dieser Daten wurde eine grobe Anordnung in Form einer Handskizze konzipiert, um ein erstes Gefühl für die Platzverhältnisse zu bekommen.

Die Konstruktion wurde in zwei funktionale Ebenen aufgeteilt: 
Eine untere Ebene zur Steuerung der Motoren und eine obere Ebene zur Steuerung der restlichen Bauteile. 
Die Ebenen wurden als separate Körper modelliert, um spätere Anpassungen zu erleichtern.

Der erste 3D-Druck diente zur Kalibrierung der Maße und Überprüfung der Passgenauigkeit. Dabei traten mehrere Fehler auf: 
Einige Schraubenlöcher waren falsch positioniert, die Inkrementalgeber lagen zu nah beieinander, und die Höhe der oberen Griffe erwies sich als zu gering, 
da die Akkumaße zunächst nur geschätzt worden waren.

Da der alte Roboter bis dahin noch softwareseitig genutzt wurde, konnte der mechanische Umbau erst nach Abschluss der entsprechenden Software-Tests (z.\,B. Implementierung des Q-Learnings) erfolgen. 
Anschließend wurde das CAD-Modell überarbeitet und an die gewonnenen Erkenntnisse angepasst.

Nach den Korrekturen konnten alle Bauteile montiert werden.
Der Aufbau des neuen Roboters war damit abgeschlossen und bildete die Grundlage für die weitere softwareseitige Integration.
Abbildung~\ref{fig:crawler_side} zeigt den Krabbelroboter in der Seitenansicht zum Projektabschluss.

\begin{figure}[H]
    \centering
    \includegraphics[width=0.8\textwidth]{crawler_side.jpeg}
    \caption{Seitliche Ansicht des Krabbelroboters zum Projektabschluss.}
    \label{fig:crawler_side}
\end{figure}

\subsection{Software}

Die Architektur der Crawler-Software, also die Strukturierung des Codes, war einer der wichtigsten Verbesserungspunkte im Vergleich zum Vorgängerprojekt (im Folgenden mit "Legacy-" bezeichnet). 

\subsubsection{Upgrade von ROS 1 zu ROS 2}

Im Vergleich zur Legacy-Codebase portierten wir das Projekt von ROS 1 zu ROS 2. ROS 2 ist zum Zeitpunkt der Arbeit die neueste Major-Version von ROS und unterstützt modernere Versionen des Ubuntu-Betriebssystems. Dadurch konnten wir das Betriebssystem auf dem Raspberry Pi von Ubuntu 18.04 auf Ubuntu 24.04 upgraden. Support für ROS 1 endete außerdem im Mai 2025, während der Laufzeit des Projekts, und die Legacy-installation hätte keine Updates mehr erhalten. Das Upgrade von ROS 1 auf ROS 2 erfordert einige grundlegendere Veränderungen an einem ROS-Projekt, da wir aber ohnehin eine vollständige Überarbeitung der Codebase anstrebten, bedeutete dies praktischerweise keinen zusätzlichen Aufwand. 

% \subsubsection{Developer Experience}

% An vorderer Stelle, sowohl in unserer Priorisierung als auch chronologisch im Projektverlauf, stand auch die Verbesserung der Developer Experience (DX), also eine bewusste Investition in einen reibungslosen Entwicklungsprozess, die mit einer geringeren Fehleranfälligkeit, erhöhtem Komfort und einer gesteigerten Entwicklungsgeschwindigkeit für uns und mögliche nachfolgende Teams einhergeht. 

% Das Legacy-Projekt verfolgte dies bereits mit einem teilweisen Umstieg von Versionskennzeichnungen im Dateinamen zu einer üblichen Versionskontrolle mit Git. Das führten wir weiter und nutzten GitHub zur einfachen Kollaboration. 

% \textbf{Synchronisierung des Quellcodes}

% Im Legacy-Projekt wurde der Programmcode direkt auf dem Raspberry Pi entwickelt, der den Roboter steuert. Wir wollten es ermöglichen, den Code lokal etwa auf dem eigenen Laptop zu editieren, um dabei eine vollwertige Entwicklungsumgebung und andere Tools verwenden zu können, für die die Leistung des Raspberry Pi nicht ausreicht. Außerdem ist dies eine logistische Erleichterung, indem der Pi nun während der Entwicklung und des Betriebs nie zwingend an einen Bildschirm angeschlossen werden muss, und mehrere Personen gleichzeitig daran arbeiten können. 

% Die Synchronisierung des Codes vom Endgerät auf den Pi erfolgt via SSH oder FTP, wobei \texttt{rsync} für eine beschleunigte Synchronisierung durch das Überspringen unveränderter Dateien genutzt wird. Es muss also eine TCP-Verbindung zum Raspberry Pi bestehen, die etwa durch ein lokales Netzwerk, unterwegs durch einen mobilen Hotspot oder auch remote durch eine Portfreigabe hergestellt werden. Eine weitere Option ist die Nutzung eines SSH Reverse Tunnels über einen öffentlichen Anbieter, wodurch lediglich der Internetzugriff durch den Pi und nicht der direkte externe Zugriff auf den SSH-Port des Pi nötig ist, was ggf. unmöglich ist. 

% Die Notwendigkeit einer Internetverbindung ist eine zusätzliche Unannehmlichkeit, die sich jedoch auszahlen dürfte, da der Pi so nicht über einen Bildschirm verfügen muss, indem das Ausführen der ROS-Startbefehle via SSH und der Zugriff auf das Webinterface via HTTP von einem externen Gerät aus geschieht; jederzeit kann stattdessen auch ein Bildschirm und eine Tastatur angeschlossen werden, um Entwicklung und Betrieb wie gewohnt offline zu ermöglichen. 

% \textbf{Vereinfachter Entwicklungszyklus mit Pixi}

% Pixi ist ein Tool zur Softwareentwicklung, das die Verwaltung von Software-Bibliotheken vereinfacht und gleichzeitig als Build-System dient. Bibliotheken und Build-Befehle werden von Pixi in einer Textdatei im Projektverzeichnis gespeichert, sodass die gesamte Entwicklungsumgebung unabhängig vom Hostsystem reproduzierbar ist.

% Der Entwicklungszyklus besteht aus mehreren Schritten, die notwendig sind, um das Programm auf dem Roboter auszuführen:
% \begin{itemize}
% 	\item Der gesamte Quellcode muss auf den Raspberry Pi synchronisiert werden.
% 	\item Die Web-Komponente wird mithilfe von npm, einem Build-Tool für JavaScript, zu einer statischen HTML-Datei gebaut. Da dies auf dem Pi ziemlich lange dauert, kann die Web-Komponente auch zuerst lokal auf dem eigenen Rechner gebaut werden, damit die resultierende HTML-Datei direkt auf den Pi synchronisiert werden kann.
% 	\item Diese HTML-Datei und der Rest des Programms werden mithilfe von ROS zu einem ausführbaren Programm gebündelt.
% 	\item Dieses Programm wird mit einem ROS-Befehl gestartet.
% \end{itemize}

% Mit Pixi konnten wir diesen mehrschrittigen Build-Prozess auf dem Roboter stark vereinfachen. Die einzelnen Schritte sind in Pixi-Befehlen wie \texttt{pixi run upload} oder \texttt{pixi run build-web} definiert, was die Arbeit mit der Codebase wesentlich erleichtert. Im Vergleich zum Vorgängerprojekt ist es außerdem nun leichter, den Build-Prozess zu dokumentieren und zu überliefern. Die Befehle zum Starten des Legacy-Programms mussten uns bei der Projektübergabe noch mündlich mitgeteilt werden, da sie in der Legacy-Codebase nicht festgehalten waren. 

% \textbf{Python Type Hints}

% Ein weiterer Schritt in der Verbesserung der Developer Experience war das Hinzufügen sogenannter Type Hints in der Python-Codebase. Als Programmiersprache mit dynamischen Typen passieren in Python schnell Fehler, wenn Variablen bezüglich ihres Typs falsch verwendet werden.

% Der folgende Python-Code produziert bei seiner Ausführung eine Fehlermeldung wegen falsch verwendeter Typen:

% \begin{minted}{python}
% def greet(s):
% 	# An dieser Stelle entsteht ein "TypeError".
% 	return "Hello " + s

% greet(1)
% \end{minted}

% Diese Fehlermeldung kann wegen der dynamischen Funktionsweise von Python jedoch erst während der Ausführung des Programms entdeckt werden. Type Hints können in einer Codebase hinzugefügt werden, um diese Art der Fehler automatisch zu finden und zu korrigieren, ohne dass eine Ausführung des Programms notwendig ist. Durch entsprechendes Tooling in der Arbeitsumgebung kann dieser Prozess nahtlos in den Entwicklungsprozess integriert werden.

% Im gezeigten Python-Code sähen Type Hints folgendermaßen aus:

% \begin{minted}{python}
% def greet(s: str) -> str:
% 	return "Hello " + s

% greet(1)
% \end{minted}

% Die hinzugefügten Type Hints geben an, dass die Variable \texttt{s} vom Typ String sein muss, sowie, dass die \texttt{greet}-Funktion einen String als Output produziert.

% Wir nutzten mypy \cite{mypy}, um die Nutzung von Type Hints strikt durchzusetzen und von ihnen im Entwicklungsprozess Gebrauch zu machen.

\subsubsection{Architektur der ROS-Nodes}

%- Architektur als Hauptverbesserungspunkt
%- wie es davor war:
%  - fehlende Sortierung zwischen Scripts, Backend-Code, Frontend-Code
%  - zwar verschiedene Nodes, aber trotzdem z.B. Hardware-Steuerung Teil der q_learning-Node etc.
%- wieso die Nodes aufgeteilt sind (anders als zuvor)
%- welche Nodes und Topics existieren, in welchem Verhältnis sie zueinander stehen:
%  - Hardware-Nodes, Web API; crawler_rl_environment, crawler_q_learning; crawler_rl_environment als Interface zwischen RL-Welt und Hardware-Welt
%  -> Diagramm

Die Verbesserung der Codebase zeigt sich vor allem in der Architektur der ROS-Nodes. In der Legacy-Codebase fanden wir initial eine Datei- und Codestruktur vor, die unseren persönlichen stilistischen Präferenzen nicht vollständig entsprach. Insbesondere in der Umsetzung einer für ROS-Applikationen idiomatischen Aufteilung in mehrere Nodes sahen wir Verbesserungspotenzial, da bspw. Code zur konkreten Hardwaresteuerung nicht von der Implementation des Q-Learning-Algorithmus abgegrenzt und teilweise mehrfach dupliziert war. Neben dem zusätzlichen Aufwand, der mit dem Weiterentwickeln einer unbekannten Codebase immer verbunden ist, waren diese stilistischen und architektonischen Differenzen mit der Legacy-Codebase, die nicht schwerwiegend oder unlösbar, aber weitreichendend waren, der Grund für unsere Entscheidung, die Crawler-Software unter Zuhilfenahme des existierenden Codes von Grund auf neu zu schreiben. 

Der neuen Architektur liegt, insbesondere im Vergleich zum Vorgängerprojekt, das Prinzip der Separation of Concerns zugrunde, also die klare Abgrenzung verschiedener Zuständigkeiten, was mit Modularität einhergeht. Dies verbessert die Entwicklung und Wartbarkeit maßgeblich, indem bei der Entwicklung einer Zuständigkeit der umliegende Kontext nur wenig detailliert betrachtet werden muss. 

% Ein Diagramm der im Folgenden beschriebenen neuen Architektur ist im Anhang auf S. \pageref{fig:architekturdiagramm} abgebildet.

\textbf{Hardware-Nodes}

Konkret bedeuteten diese Prinzipien für uns zunächst, die Steuerung einzelner Hardwarekomponenten wie der Inkrementalgeber oder der Motoren zur Zuständigkeit je einer ROS-Node zu machen. In der frühen Entwicklungsphase verwendeten wir, um die konzipierte Funktionsweise der Hardwaresteuerung und später des Zusamenspiels mit dem Webinterface auszuprobieren, eine simple blinkende LED ("Blinker") als Proof of Concept. Im Folgenden ist die Konzeption der Hardware-Nodes an diesem Beispiel, der Node \texttt{crawler\_blinker}, erläutert. Da die LED an einem GPIO-Pin ("General Purpose I/O") des Raspberry Pi angeschlossen ist, übergeben wir den konkreten Pin als Parameter, der in der ROS-Launchdatei noch nach dem Bauen des ROS-Pakets je nach gewünschtem Hardwareaufbau angepasst werden kann. Beim Start stellt die Node außerdem anhand einer Umgebungsvariable fest, ob als zugrundeliegende Implementation die Steuerung der physischen LED über die GPIO-Pins oder eine Mock-LED verwendet werden soll, welche sich nach außen hin wie die reale LED verhält, jedoch keine angeschlossene Hardware voraussetzt, und damit bspw. zur lokalen Entwicklung des Webinterfaces geeignet ist. Für die Operationen, die die LED-Node exponieren soll, werden Subscriptions zu den Topics \texttt{/crawler/blinker/toggle} und \texttt{/crawler/blinker/write} erstellt. Wird dort eine Message mit einem \texttt{Empty}- bzw. \texttt{Boolean}-Wert empfangen -- bei der Motorsteuerung wären das bspw. Zahlenwerte für eine relative Bewegung --, wird die Hardware entsprechend angesteuert. Der aktualisierte Hardwarezustand wird auf dem Topic \texttt{/crawler/blinker/state} veröffentlicht -- bei reinen Sensoren wie den Inkrementalgebern geschieht dies bei jedem neuen Messwert. Die Nodes \texttt{crawler\_motors} und \texttt{crawler\_encoders} zur Steuerung der Motoren bzw. Inkrementalgeber funktionieren nach demselben Prinzip.

\textbf{Web-API-Node}

Zur manuellen Hardwaresteuerung und Anzeige der Hardwarezustände im Webinterface existiert die Node \texttt{crawler\_web\_api}, die hauptsächlich als Adapter zwischen ROS und dem Web fungiert, indem sie den Zugriff auf die ROS-Topics über entsprechende HTTP-Endpoints bereitstellt. So gibt es bspw. den Endpoint \texttt{/api/manual/blinker/toggle} zum Versenden einer Message auf der ROS-Topic \texttt{/crawler/blinker/toggle}. Für die Topics, auf denen die Hardware-Nodes ihren Zustand veröffentlichen, muss ein anderer Mechanismus verwendet werden, da der Datenfluss hier vom Server zum Client verläuft. Bereits im Legacy-Projekt wurden dafür WebSockets verwendet, da sie es dem Server ermöglichen, über eine offen gehaltene TCP-Verbindung aktiv Nachrichten an den Client zu senden. Jedoch wurde dabei nur die Information gesendet, dass ein Update vorliegt, sodass der Client mit einem regulären HTTP-Request reagieren kann, in diesem Fall, um eine gerenderte Diagramm-Grafik des Lernprozesses vom Server zu laden. Das hat den Nachteil, dass durch den zusätzlichenen Request eine Verzögerung entsteht. Es ist außerdem schwieriger, verschiedene Arten von Daten zu übertragen, was vermutlich der Grund dafür ist, dass nur ein Diagramm angezeigt wird, und nicht, wie wir es vorhaben, bspw. weitere Hardwarezustände. Um das zu lösen, übertragen wir die Daten direkt über die WebSocket-Verbindung in einem numerischen Format, um sie anschließend im Webinterface dynamisch in Diagrammen und anderen Visualisierungen rendern zu können. Um die Anzahl offener Verbindungen zu begrenzen, aggregieren wir die Daten mehrerer Topics in einem WebSocket-Endpoint; so werden etwa bei einer Nachricht, die in der Web-API-Node von einer der Hardwarezustand-Topics empfangen wird, die entsprechenden Felder in einer Datenstruktur, die den gesamten Hardwarezustand darstellt, aktualisiert, und anschließend der gesamte Hardwarezustand über den WebSocket-Endpoint gesendet. 

\textbf{Reinforcement-Learning-Nodes}

\textbf{RL-Environment-Node:} Der Teil der Applikation, der für das Reinforcement Learning zuständig ist (siehe Abschnitt~\ref{sec:reinforcement_learning}), besitzt ebenso eine sehr modulare Struktur. Zunächst gibt es die Node \texttt{crawler\_rl\_environment}, die den Reinforcement-Learning-Prozess überwacht und verwaltet. Außerdem übersetzt sie, entsprechend ihrer Benennung, die über die Topics gelesenen Hardwarezustände in ein für den RL-Algorithmus passendes Format eines Environments im Sinne des RL-Frameworks; ebenso wird der Reward anhand der Inkrementalgeber-Daten bestimmt. Der Reinforcement-Learning-Zyklus wird von der Environment-Node ausgehend gesteuert: Nachdem sie eine Nachricht zum Starten empfangen hat, sendet sie das erste Environment (und keinen Reward) auf \texttt{/crawler/rl/state\_reward} und wartet auf die Antwort der RL-Node auf \texttt{/crawler/rl/action}. Entsprechend der gewünschten Aktion werden Nachrichten zur Hardwaresteuerung gesendet, und das nächste Environment und der nächste Reward werden in einem fortlaufenden Prozess gesendet.
% TODO: "Reinforcement-Learning-Zyklus" bessere Formulierung?

\textbf{RL-Node:} Der Reinforcement-Learning-Algorithmus ist in der RL-Node implementiert, welche die oben genannten Topics Environment und Reward von der Environment-Node empfängt bzw. eine Aktion sendet. Bisher ist nur ein konkreter Algorithmus implementiert, in Zukunft sind aber auch verschiedene Implementationen denkbar; da die Schnittstelle mit der Environment-Node abstrakt gehalten ist, lässt sich die RL-Node also beliebig durch andere Implementationen ersetzen. Daher wird die RL-Node erst bei Bedarf z.\,B. über das Webinterface von der Web-API-Node mit den eingestellten Parametern gestartet.

Um den Crawler als Demonstrationsroboter geeignet zu machen, möchten wir den internen Zustand des RL-Algorithmus anschaulich visualisieren. Dazu senden die RL-Environment-Node und die RL-Node ihren internen Zustand, der von der Web-API-Node aggregiert und auf einem WebSocket-Endpoint veröffentlicht wird. 

\subsubsection{Webinterface}

Bereits das unmittelbare Vorgängerprojekt hatte die Verbesserung des Crawlers hinsichtlich seiner Eignung zu Demonstrationszwecken zum Ziel, indem ein simples Webinterface zur manuellen Steuerung sowie zum Starten und Überwachen des Reinforcement Learning implementiert wurde. Wir führten diesen Gedanken weiter, indem wir ein neues Webinterface von Grund auf neu entwickelten. Da die Verwendung des Crawlers zu Demonstrationszwecken die für unser Projekt am relevantesten erscheinende Anwendung ist, investierten wir hierbei auch über reines Funktionieren hinaus in eine flüssige und ansehnliche Benutzererfahrung. 

Neben einer Home-Seite mit einführenden Informationen zum Crawler und Links zur GitHub-Seite hat das Webinterface zwei Hauptfunktionen. Zunächst ist das die manuelle Kontrolle (\textbf{Manual Control}) aller Hardwarekomponenten, d. h. Steuerung der Motoren und Darstellung der von den Inkrementalgebern eingelesenen Daten, was bspw. zum Debuggen in einem früheren Stadium des Projekts sehr praktisch war und weiterhin sein wird. Zur Veranschaulichung sind die UI-Elemente den physischen Komponenten visuell nachempfunden, und der aktuelle Zustand wird sowohl numerisch als auch durch eine entsprechende Animation der abgebildeten Komponenten dargestellt. Da bei einer Demonstration der Bildschirm womöglich auf eine solche Weise projiziert wird, dass dem Mauszeiger, sofern es einen gibt, nicht leicht zu folgen ist, werden Interaktionen mit den Bedienflächen durch ein rotes Aufblinken klar verdeutlicht. Das \textbf{RL Control} bietet eine Benutzeroberfläche zum Konfigurieren und Starten des Reinforcement Learning zur Steuerung des Roboters. Im laufenden Betrieb des RL-Algorithmus werden außerdem in Echtzeit Informationen zum aktuellen internen Zustand angezeigt, die möglichst anschaulich visualisiert werden, sodass man als Publikum oder Forschende etc. die Funktionsweise des Demonstrationsroboter leicht nachvollziehen kann. So ist erstens der Zustand des RL-Environments mit dem letzten Reward (in farblicher Kennung entsprechend dem Vorzeichen), der zuletzt gewählten Aktion und einem Diagramm visualisiert, das den Lernfortschritt, d. h. die insgesamt zurückgelegte Strecke, über die Zeit darstellt und ebenso in Echtzeit angepasst wird. Zweitens wird die Q-Table als interner Zustand der RL-Node offengelegt. Dabei werden die Tablellenzellen entsprechend ihrem Wert eingefärbt, um die Gewichtung zu verdeutlichen; die Anpassung der Q-Table, die nach jedem Schritt vorgenommen wird, wird durch eine kurze Hervorhebung zusätzlich angezeigt. 

Eine später hinzugefügte Bedienoberfläche ist die Möglichkeit, in die Auswahl der nächsten Aktion durch den Q-Learning-Algorithmus manuell einzugreifen, was es er\-möglicht, den Lernprozess zur besseren Nachvollziehbarkeit schrittweise ablaufen zu lassen. Zuletzt ist noch die Möglichkeit essenziell, trainierte Modelle, d. h. den Inhalt einer Q-Table, exportieren und bei Starten des Q-Learning erneut importieren zu können. Dies geschieht in einem serialisierten Textformat, das beliebig lokal gespeichert werden kann. Diese Option erlaubt es, einen erfolgreichen Lernvorgang festzuhalten und zu einem späteren Zeitpunkt zu nutzen. 

Für die Umsetzung des Backends verwenden wir den simplen Python-Webserver \emph{Flask} \cite{flask}, der auch bereits im Vorgängerprojekt verwendet wurde. Das Webinterface ist mit der UI-Library \emph{React} \cite{react} implementiert, welche einen modernen komponentenbasierten Workflow bietet und von Grund auf reaktiv ist, was die Anzeige der Echtzeitdaten maßgeblich erleichtert. Für eine moderne UI-Entwicklung verwenden wir \emph{Tailwind CSS} \cite{tailwindcss} als Styling-Framework, \emph{React Router} \cite{reactrouter} als React-Framework und \emph{Chart.js} \cite{chartjs} für dynamische Diagramme.

% Im Anhang, S. \pageref{fig:webinterface_screenshots}f, sind einige Screenshots abgebildet.
	\section{Ergebnisse}
\label{sec:realisierung}

% Dieses Kapitel gibt einen kompakten Überblick über die praktische Umsetzung des neuen Krabbelroboters. Die wichtigsten Arbeitsschritte, Herausforderungen und \linebreak Lösungswege von der Übernahme des Legacy-Projekts bis zur Entwicklung der neuen Hardware werden dargestellt.

% \subsection{Übernahme des Legacy-Projekts}

% Erstes Ziel war zu Projektbeginn, das Vorgängerprojekt zu reaktivieren und zum Laufen zu bekommen. Entsprechende Versuche scheiterten jedoch an unterschiedlichen Faktoren: Der Quellcode des Legacy-Programms war zwar öffentlich auf GitHub synchronisiert, allerdings ohne jegliche Anleitung bezüglich des Programmstarts, benötigter Software-Bibliotheken oder Veränderungen an der Installation des Betriebssystems. Nach mehreren Stunden des Debuggens war der beste Zustand, den wir erreichen konnten, das Programm mit ROS 1 in einem Docker-Image auszuführen. Da die isolierte Container-Umgebung allerdings scheinbar immer noch nicht den nötigen Startbedingungen entsprach, stürzte das Legacy-Programm weiterhin nach Start sofort ab. Nachdem die Arbeit eines ganzen Tages ein so enttäuschendes Ergebnis geliefert hatte, entschieden wir, das Programm von Grund auf neu zu schreiben und dabei den Fokus auf gute Dokumentation des Projekts und eine gute Developer Experience zu legen, um uns selbst sowie möglichen Nachfolgern des Projekt die Arbeit softwareseitig zu erleichtern.

% \subsection{Probleme mit dem Raspberry Pi}

% Im Zuge des Neuschreibens der Codebase sollte auch die auf dem Raspberry Pi installierte Ubuntu-Version angehoben werden. Als wir allerdings versuchten, den Microcontroller auf dem alten Roboter zu starten, hing dieser stets in einer Boot-Schleife, startete also nicht das Betriebssystem. Nach wiederum einem Tag des Debuggens mit verschiedenen Betriebssystemen, Speichermedien, Netzteilen und Monitorkabeln war mit dem alten Raspberry Pi immer noch kein Betriebbsystem zu starten. Wir kamen zu dem Schluss, dass der spezifische Raspberry Pi auf dem alten Roboter zu Schaden gekommen sein musste, seit wir das letzte Mal damit gearbeitet hatten, vermutlich im Transport. Freundlicherweise konnte uns Prof. Dr. Ihme ein Ersatzmodell aus dem Inventar der Technischen Hochschule Mannheim beschaffen, sodass wir zeitig an dem Projekt weiterarbeiten konnten.

\subsection{Hardware}

Ausgehend von der theoretischen und praktischen Analyse des bestehenden Roboters (siehe Abschnitt~\ref{sec:konzeption}) wurde ein neues mechanisches Konzept für das Chassis des Krabbelroboters 
entwickelt. Ziel war es, die Komponenten platzsparend, gewichtsoptimiert und modular anzuordnen um die Zugänglichkeit zu erweitern.

Zunächst erfolgte eine Recherche der Maße der Bauteile. Auf Basis dieser Daten wurde eine grobe Anordnung in Form einer Handskizze konzipiert, um ein erstes Gefühl für die Platzverhältnisse zu bekommen.

Das daraus abgeleitete 3D-Modell wurde mithilfe von Autodesk Fusion 360 erstellt. Die Konstruktion wurde in zwei funktionale Ebenen aufgeteilt: 
eine untere Ebene zur Steuerung der Motoren und eine obere Ebene zur Steuerung der restlichen Bauteile. 
Die Ebenen wurden als separate Körper modelliert, um spätere Anpassungen zu erleichtern.

Der erste 3D-Druck diente zur Kalibrierung der Maße und Überprüfung der Passgenauigkeit. Dabei traten mehrere Fehler auf: 
Einige Schraubenlöcher waren falsch positioniert, die Inkrementalgeber lagen zu nah beieinander, und die Höhe der oberen Griffe erwies sich als zu gering, 
da die Akkumaße zunächst nur geschätzt worden waren.

Da der alte Roboter bis dahin noch softwareseitig genutzt wurde, konnte der mechanische Umbau erst nach Abschluss der entsprechenden Software-Tests (z.\,B. Implementierung des Q-Learnings). 
Anschließend wurde das CAD-Modell überarbeitet und an die gewonnenen Erkenntnisse angepasst.

Der zweite Druckvorgang verlief erfolgreich: Die Bauteile passten wie geplant, alle Komponenten konnten montiert werden. 
Der Aufbau des neuen Roboters war damit abgeschlossen und bildete die Grundlage für die weitere softwareseitige Integration.
Abbildung~\ref{fig:crawler_side} zeigt den Krabbelroboter in der Seitenansicht zum Projektabschluss.

\begin{figure}[H]
    \centering
    \includegraphics[width=0.8\textwidth]{crawler_side.jpeg}
    \caption{Seitliche Ansicht des Krabbelroboters zum Projektabschluss.}
    \label{fig:crawler_side}
\end{figure}
 % Ergebnisse, JuFo
	\section{Diskussion}
\label{sec:diskussion}

Im folgenden Abschnitt wird kritisch auf die Umsetzung zurückgeblickt. Fehler und fehlgeschlagene Zielsetzungen werden analysiert, und es werden Erkenntnisse für eine Fortführung der Arbeit formuliert.

\subsection{Fehlerquellen und methodische Grenzen}

\textbf{Mechanische Fehler}

Im mechanischen Bereich kam es vor allem durch gemachte Fehler im CAD-Design (und die begrenzte Maßgenauigkeit des 3D-Druckverfahrens) zu Problemen. Mehrere Teile mussten nachgedruckt werden, da z.\,B. Bohrungen oder Bauteilabstände nicht korrekt passten oder Fehler im Design gemacht wurden. 
Außerdem wurde PLA-Filament für die Verbindung von Bein und Fuß untereinander und mit der Basis verwendet, was sich negativ auf die Stabilität auswirken könnte. Systematische Tests dazu konnten jedoch leider nicht durchgeführt werden.

\subsection{Bewertung der Umsetzung}

\textbf{Bewertung der mechanischen Umsetzung}

Die angestrebte Modularität konnte umgesetzt werden: Alle zentralen Komponenten lassen sich in akzeptabler Zeit austauschen. Das hat sich besonders bei der Arbeit mit der Stromversorgung als hilfreich erwiesen, z.\,B. beim Aufladen des Akkus. Ein Nachteil besteht derzeit noch im Kabelmanagement, das durch die Offenheit und Modularität des Design nicht umgesetzt wurde.

Im Vergleich zum alten Roboter stellt die neue Konstruktion eine deutliche Verbesserung dar. Dort musste beim Abbau bspw. die gesamte obere Ebene zersägt werden, um an einige Kabel zu gelangen.

Das äußere Erscheinungsbild wirkt funktional und aus unserer Sicht auch ansprechend, wobei dies natürlich subjektiv ist. Der Schwerpunkt des Roboters liegt tief, was sich positiv auf die Stabilität auswirken sollte. Systematische Tests konnten aber auch hier leider nicht durchgeführt werden. Eine Verkleinerung der Grundfläche wäre künftig sinnvoll, da aktuell relativ viel Material verbraucht wird, was auch die Druckzeit merklich erhöht. 

Insgesamt konnte das Gewicht des Roboters durch die neue Konstruktion aber reduziert werden. Die genaue Reduktion liegt bei ca. \qty{48,86}{\percent} (von \qty{1,32}{\kilo\gram} auf \qty{0,675}{\kilo\gram}).

% Die optionalen Gestaltungskriterien wurden nicht umgesetzt. 

\pagebreak
\textbf{Bewertung der softwareseitigen Arbeit}

Die neue Programm wurde in der Laufzeit des Projekts auf denselben Funktionsstand des Legacy-Programms gebracht, darüber hinaus wurden nur wenige Funktionen hinzugefügt. Allerdings wurde das gesamte bestehende Programm in nahezu jeder Hinsicht verbessert:

\begin{itemize}
	\item Das Webinterface wurde mit modernen Bibliotheken reimplementiert und im Zuge dessen visuell ansprechender gestaltet.
	\item Der Entwicklungsprozess wurde wesentlich reibungsloser und angenehmer gestaltet. Zukünftige Entwicklung am Roboter sollte wesentlich schneller verlaufen können.
	\item Der gesamte Entwicklungsprozess sowie die benötigte Entwicklungsumgebung sind gut dokumentiert und leicht zu verwalten. Dies wird für zukünftige Projekte den Projektanfang stark erleichtern.
	\item Das neue Programm wurde wesentlich modularer gestaltet, was künftige Änderung\-en und Additionen erleichtert.
\end{itemize}

Auch wenn der Projektstand zunächst im Vergleich zum Vorgängerprojekt nicht fortgeschritten zu sein scheint, wurde doch viel an der Software weiterentwickelt, und die Ergebnisse können sich durchaus zeigen lassen.

Dennoch wäre Weiterentwicklung über das Vorgängerprojekt hinaus im Sinne einer tatsächlichen Fortbewegung wünschenswert gewesen. Wir konnten zwar durch menschliche Anleitung während des Trainingsprozesses ein Modell erzeugen, das beim Aus\-führen für eine zuverlässige Fortbewegung sorgt, dies erfüllt jedoch nicht ausreichend den Anspruch eines selbstlernenden Roboters. Ideen zu Projektanfang wie verschiedene Algorithmen zur Steuerung des Roboters oder Aufzeichnung und Simulation von Bewegungen und Fortschritt wurden in der gegebenen Zeit nicht realisiert.

\subsection{Erkenntnisse und Deutung}

Rückblickend wurde der Aufwand für die mechanische Umsetzung deutlich unterschätzt. Insbesondere die CAD-Konstruktion, das mehrfache Drucken, der Abbau des alten Roboters und der anschließende Aufbau des neuen Systems nahmen mehr Zeit in Anspruch als geplant. Das Zeitmanagement hätte entsprechend besser abgestimmt werden müssen, um Engpässe in der finalen Phase zu vermeiden.

Triviale Hindernisse wie ein defekter Raspberry Pi oder das Starten des Legacy-\hspace{0px}Programms waren ebenfalls für Verzögerungen im Projektablauf verantwortlich. Diese Hindernisse wurden anfangs auch nicht bedacht, obwohl man mit Problemen solcher Art hätte rechnen können. Sie waren ebenfalls Grund, dass nicht alle vorgesehenen Ziele rechtzeitig erreicht werden konnten. In Zukunft sollte versucht werden, solche Probleme frühzeitig zu erkennen, schneller zu diagnostizieren und geschickter zu lösen, anstatt wertvolle Tage der gemeinsamen Arbeit daran zu verschwenden.

\subsection{Ausblick und offene Fragen}

Zentrales Ziel weiterführender Projekte sollte das Erreichen einer tatsächlichen Fortbewegung des Roboters sein. Weitere konkrete Aspekte sind von Interesse:

\textbf{Mechanischer Aufbau}

Für eine zukünftige mechanische Weiterentwicklung wäre ein strukturierterer Innenaufbau sinnvoll, um Ordnung und Zugänglichkeit weiter zu erhöhen. Auch die angesprochenen fehlenden Tests fanden bisher nicht statt, obwohl diese helfen könnten, Schwachstellen besser zu erkennen. Darüber hinaus wäre eine umfassendere strukturelle Analyse des Roboters hilfreich, um mechanische Belastungen gezielter beurteilen und optimieren zu können.

\textbf{Software-Algorithmen}

Softwareseitig wäre es interessant, verschiedene Machine-Learning-Algorithmen genauer zu vergleichen. Es könnte etwa der existierende Q-Learning-Algorithmus und eine Implementation eines neuronalen Netzwerks verglichen werden. Aufgrund der mangelnden Verfügbarkeit der Motoren des Roboters konnte der Vergleich in der Zeit unseres Projekts nicht durchgeführt werden. \\ Fernere Möglichkeit zur Weiterentwicklung wäre etwa eine automatische Steuerung von Versuchsreihen nach konfigurierbaren Parametern. 

% \section*{Danksagung}
% \label{sec:schlussteil}
% \addcontentsline{toc}{section}{Danksagung}

% Wir bedanken uns bei den Leitern unseres Hector-Kurses, Dr. Rolf Piffer und Inka Briese, welche schon lange mit viel Aufmerksamkeit unseren Kurs betreuen. Wir danken auch unserem Kooperationspartner Prof. Dr. Thomas Ihme der Technischen Hochschule Mannheim, der uns entlang des Projekts stets mit Engagement unterstützte, sowie dem Ehepaar Hector, welches dieses Seminar zur Begabtenförderung ins Leben gerufen hat. Jede der genannten Personen hat maßgebend dazu beigetragen, uns dieses Kooperationsprojekt zu ermöglichen.

	% \section*{Anhang}
% \label{sec:anhang}
% \addcontentsline{toc}{section}{Anhang}

% Der Quellcode für die Crawler-Software, die CAD-Modelle und diese Arbeit sind auf GitHub veröffentlicht: 
% \texttt{https://github.com/hector-crawler} % Andere Stelle hinzufügen?

% \label{fig:webinterface_screenshots}

% \begin{figure}[H]
%     \includegraphics[width=0.5\textwidth]{photos/webinterface_home.png}
%     \caption{Webinterface-Seite \texttt{/home} mit einleitenden Informationen und weiterführenden Links}
% \end{figure}

% \begin{figure}[H]
% 	\includegraphics[width=0.7\textwidth]{photos/webinterface_manualControl.png}
%     \caption{Webinterface-Seite \texttt{/manualControl} zur manuellen Kontrolle und Anzeige der Hardwarekomponenten}
% \end{figure}

% \begin{figure}[H]
% 	\includegraphics[scale=0.22]{photos/webinterface_rlControl_1.png}
% 	\caption{Webinterface-Seite \texttt{/rlControl} zur Konfiguration der Parameter des Q-Learning}
% \end{figure}

% \begin{figure}[H]
% 	\includegraphics[width=\textwidth]{photos/webinterface_rlControl_2.png}
% 	\caption{Webinterface-Seite \texttt{/rlControl} nach dem Starten, mit Visualisierungen des RL-Environments (links) und der Q-Table (rechts)}
% \end{figure}

% \label{fig:architekturdiagramm}
% \includepdf[pages={-}]{photos/crawler-architecture.pdf}

% Sämtliche Photos und Graphiken dieses Dokuments wurden, soweit nicht ausdrücklich anders gekennzeichnet, von Mitgliedern des Teams erstellt.

% %%%%

\section*{Quellen}
\addcontentsline{toc}{section}{Quellen}
\bib
\appendix

% \newcommand{\selbststaendigkeitserklaerung}[1]{
% 	\newpage
% 	\section*{Selbstständigkeitserklärung}
% 	Hiermit versichere ich, {#1}, dass ich diese Arbeit unter der Beratung durch Herrn Prof. Dr. Ihme selbstständig verfasst habe und keine anderen als die angegebenen Quellen und Hilfsmittel benutzt wurden, sowie Zitate kenntlich gemacht habe. \\[2cm]

% 	Ort, Datum \hfill \rule{5cm}{0.4pt} \\
% 	\hfill (Unterschrift)
% }

% \selbststaendigkeitserklaerung{Gregor Niehl}
% \selbststaendigkeitserklaerung{Jonathan Kraus}
% \selbststaendigkeitserklaerung{Pascal Roth}

\end{document}
